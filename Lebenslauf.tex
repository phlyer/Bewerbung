

\documentclass[sans,11pt,a4paper]{moderncv}

\linespread {1.125}


% moderncv themes
%\moderncvtheme[blue]{casual}                 % optional argument are 'blue' (default), 'orange', 'red', 'green', 'grey' and 'roman' (for roman fonts, instead of sans serif fonts)
\moderncvtheme[blue]{casual}                % idem

% character encoding
\usepackage[utf8]{inputenc}                   % replace by the encoding you are using
\usepackage[ngerman]{babel}

% adjust the page margins
\usepackage[scale=0.85]{geometry}
\recomputelengths                             % required when changes are made to page layout lengths
\setlength{\hintscolumnwidth}{2.75cm}




% personal data
\firstname{Philipp}
%\middlename{}
\familyname{Lindner}
%\title{Resumé title (optional)}               % optional, remove the line if not wanted
\address{Ruschgraben 38}{76139 Karlsruhe}    % optional, remove the line if not wanted
\mobile{+49 162 89\,33\,00\,8}                    % optional, remove the line if not wanted
%\phone{+49 30 68\,81\,12\,30}                      % optional, remove the line if not wanted
%\fax{+49 6201 80\,66\,98}
\email{pl@xq0.net}                          % optional, remove the line if not wanted
%\email{lindner@inwi.org}                      % optional, remove the line if not wanted
%\extrainfo{www.squealj.net} % optional, remove the line if not wanted
\photo[64pt]{picture}                         % '64pt' is the height the picture must be resized to and 'picture' is the name of the picture file; optional, remove the line if not wanted
\quote{* 28. September 1989}                 % optional, remove the line if not wanted

%\nopagenumbers{}                             % uncomment to suppress automatic page numbering for CVs longer than one page




\makeatletter
%\AtBeginDocument
%{
% reverse the name and photo
\renewcommand*{\makecvtitle}{%
  \recomputecvlengths%
  \makecvfooter%
  % define optional picture
  \newbox{\makecvtitlepicturebox}%
  \savebox{\makecvtitlepicturebox}{%
    \ifthenelse{\isundefined{\@photo}}%
      {
        \@initializelength{\makecvtitlepicturewidth}% Damit Länge bekannt bei Name
        \settowidth{\makecvtitlepicturewidth}{0pt}%
      }%
      {%
       \setlength\fboxrule{\@photoframewidth}%
       \ifdim\@photoframewidth=0pt%
         \setlength{\fboxsep}{0pt}\fi%
       {\color{color1}\framebox{\includegraphics[width=\@photowidth]{\@photo}}}}
        \@initializelength{\makecvtitlepicturewidth}% Damit Länge bekannt bei Name
        \settowidth{\makecvtitlepicturewidth}{\usebox{\makecvtitlepicturebox}}%
      }%
  % end define optional picture

  % name
%    \parbox[b]{\textwidth-\makecvtitlepicturewidth}{%
 {   \raggedright\namefont{\color{color2}\MakeLowercase{\@firstname}}\color{color1}\MakeLowercase{\@familyname}}%\lastname -> error
    \hfill\usebox{\makecvtitlepicturebox}%
%    \parbox[b]{\textwidth-\makecvtitlepicturewidth}{%
%   \raggedleft\namefont{\color{color2!50}\@firstname} {\color{color2}\@lastname}}%\familyname
\\[-.35em]%
  {\color{color2}\rule{\textwidth}{.25ex}}%
  % optional title
  \ifthenelse{\equal{\@title}{}}{}{\\[1.25em]\null\hfill\titlestyle{\@title}}\\[2.5em]%
  % optional quote
  \ifthenelse{\isundefined{\@quote}}%
    {}%
    {{\null\hfill\begin{minipage}{\quotewidth}\centering\quotestyle{\@quote}\end{minipage}\hfill\null\\[2.5em]}}%
  \par}%
%}% AtBeginDocument ende
\makeatother


\begin{document}
\vspace{-0.5cm}
\makecvtitle



\section{Ausbildung}
\cventry{05.2014--03.2017}{Master}{Karlsruher Institut für Technologie}{Informationswirtschaft}{Vertiefung in den Bereichen Entwicklung betrieblicher Informationssysteme, Kommunikation und Datenerhaltung, Service Management und Computing, Thesis: \glqq Öffnung und Anbindung von Strommärkten in Europa – Systemmodell zur Szenarioanalyse\grqq}{Abschluss: Master of Science, Notenschnitt: 1,6}
\cventry{08.2014--01.2015}{Ausland}{Chalmers University of Technology, Göteborg, Schweden}{Computer Science}{}{}
\cventry{10.2010--04.2014}{Bachelor}{Karlsruher Institut für Technologie}{Informationswirtschaft}{Vertiefung u.a. in den Bereichen Algorithmen, Customer Relationship Management, Thesis: \glqq Entwicklung eines individualisierbaren Regelschemas für AAL-Monitoringsysteme\grqq\ am Forschungszentrum Informatik (FZI) Karlsruhe}{Abschluss: Bachelor of Science, Notenschnitt: 2,1}
\cventry{09.2000--07.2009}{Abitur}{Carl-Friedrich-Gauß-Gymnasium Hockenheim}{}{}{Notenschnitt: 2,2}%Leistungskurse: Deutsch, Englisch, Gemeinschaftskunde, Physik
%\cventry{09.1997--07.2000}{Grundschule}{Alte Schule Ketsch}{}{}{}
\closesection
\vspace{0.5cm}



\section{Berufserfahrung}
%\subsection{beruflich}
\cventry{05.2016--04.2017}{Werkstudent und Thesis}{EXXETA Consulting and Technologies}{Energy Trading \& Risk Management}{}{}
\cventry{03.2014--07.2014}{Praktikum und Werkstudent}{ec4u expert consulting ag}{Strategy \& Business Consulting für CRM, BI und Integration}{}{}
\cventry{04.2012--03.2014}{Wissenschaftliche Hilfskraft}{Institut für Informationswirtschaft und Marketing (IISM)}{Tutorien für Einführung in die Informationswirtschaft und Customer Relationship Management}{}{}
%\cventry{Seit 01.2012}{Aushilfe}{Buchecke Oftersheim}{IT und EDV Administration}{}{}
\cventry{11.2011--03.2012}{Wissenschaftliche Hilfskraft}{Studienbüro Karlsruher Institut für Technologie}{Verwaltungstätigkeiten}{}{}
\cventry{10.2009--07.2010}{Zivildienst}{SRH Berufliche Rehabilitation, Heidelberg}{Betreuung von Rehabilitanten, Aufbau IT-Infrastruktur, Veranstaltungsorganisation}{}{}
\cventry{02.2006}{Praktikum}{Freudenberg Service KG, Weinheim}{Elektrotechnik, Informationstechnik und Betriebswirtschaftslehre}{}{}                % arguments 3 to 6 are optional
\closesection
\vspace{0.5cm}


%\pagebreak{}

\section{Ehrenamtliche Tätigkeiten}
\cventry{Seit 04.2015}{Vorstand im Rotaract Club Hockenheim Schwetzingen}{Aufgaben: Vereinsgründung, Entwurf von Satzung und Verfassung sowie regionale und globale soziale Projekte}{}{}{}
\cventry{11.2012--07.2014}{Studienkommission Informationswirtschaft}{Karlsruher Institut für Technologie}{\textit{Aufgaben: Diskussion und Handlungsempfehlung zum Studiengang Informationswirtschaft}}{}{}
\cventry{Seit 10.2010}{Forum Informationswirtschaft e.V.}{Aufgaben: Organisation der Orientierungsphase 2012 für die Erstsemester sowie Durchführung verschiedener kleinerer Veranstaltungen}{}{}{}% arguments 3 to 6 are optional
\cventry{09.2007--07.2009}{Stufensprecher der Oberstufe}{Aufgaben: Organisation von Veranstaltungen, Erstellen der Abizeitung, Betreuung der Finanzen u.v.m}{}{}{}
\cventry{05.2006--09.2010}{Gruppenleiter}{Aufgaben: Betreuung einer Jugendgruppe und Planung von Jugendfreizeiten im In- und Ausland}{}{}{}
\cventry{09.2000--07.2009}{Schülermitverantwortung}{Aufgaben: Organisation von Konflikttagen und Informationsveranstaltungen zu Toleranz, Rassismus u.v.m.}{Carl-Friedrich-Gauss-Gymnasium Hockenheim}{}{}
\cventry{Seit 10.1997}{Deutsche Pfadfinderschaft St. Georg}{Stamm Don Bosco Ketsch}{}{}{} % arguments 3 to 6 are optional
%\cventry{2007--2008}{Business @ School der Boston Consulting Group}{Aufgaben: Analyse von Unternehmen, Gründung eines fiktiven Unternehmens}{Schulsieger}{}{}
\closesection
\vspace{0.5cm}


\section{Fortbildungen und Sonstiges}
\cventry{03.2012--09.2012}{Tutorenprogramm am Karlsruher Institut für Technologie}{Methoden und Strategien zur Unterstützung von studentischem  Lernen}{}{}{}
\cventry{01.2010}{Einwöchiges Anwender-Seminar SAP ERP Finanzwesen}{Themen: Organisationsstrukturen und Stammdaten, Geschäftspartnerbuchhaltung, Hauptbuchhaltung, Reporting}{}{}{}
\cventry{09.2007--07.2008}{Business @ School der Boston Consulting Group}{Aufgaben: Analyse von Unternehmen, Gründung eines fiktiven Unternehmens}{Schulsieger}{}{}


\closesection
\vspace{0.5cm}



\section{Sprachen}
\cvcomputer{Deutsch}{Muttersprache}{Englisch}{fließend in Wort und Schrift}
\closesection
\vspace{0.5cm}


\section{Kenntnisse}
\cvcomputer{IT}{Microsoft Windows, Microsoft Office}{Entwicklung}{XML, HTML, SQL, Java, JavaScript}
\cvcomputer{SAP}{ERP Finanzwesen}{Modellierung}{UML, BPMN, Petri-Netze}
\cvcomputer{Grafik \& Layout}{LaTeX, Adobe Photoshop, Illustrator und InDesign}{IDE}{Eclipse, Netbeans, Altova XMLSpy, Arduino}
\cvcomputer{Ontologien}{OWL, SWRL}{Projekt}{Projektron BCS}
\closesection
\vspace{0.5cm}


\section{Interessen}
\cvcomputer{Sport}{Wintersport, Wandern, Radfahren}{Kultur}{Pfadfinder, Reisen, Kochen, Lesen}
\cvcomputer{Technik}{Heimkino, GPS Tracking}{}{}
\closesection{}                   % needed to renewcommands


\begin{flushleft}
Stand: \today
\end{flushleft}


\end{document}


%% end of file `template_en.tex'.
